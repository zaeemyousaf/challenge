% Created 2020-11-29 Sun 22:51
% Intended LaTeX compiler: pdflatex
\documentclass[11pt]{article}
\usepackage[utf8]{inputenc}
\usepackage[T1]{fontenc}
\usepackage{graphicx}
\usepackage{grffile}
\usepackage{longtable}
\usepackage{wrapfig}
\usepackage{rotating}
\usepackage[normalem]{ulem}
\usepackage{amsmath}
\usepackage{textcomp}
\usepackage{amssymb}
\usepackage{capt-of}
\usepackage{hyperref}
\author{submitted to: Amina Khan (DS assignment 2)}
\date{Workout of Time Complexity}
\title{ZAEEM YOUSAF (l19-1196 3E2)}
\hypersetup{
 pdfauthor={submitted to: Amina Khan (DS assignment 2)},
 pdftitle={ZAEEM YOUSAF (l19-1196 3E2)},
 pdfkeywords={},
 pdfsubject={},
 pdfcreator={Emacs 26.1 (Org mode 9.1.9)}, 
 pdflang={English}}
\begin{document}

\maketitle
\tableofcontents

\section{Q1/B Time complexity of iterative mirror}
\label{sec:orgee8c8ae}
ans: \textbf{O(n)}
\subsection{Detail}
\label{sec:orgf5bb0a9}
all right nodes were pushed and poped O(log n)
each time, a poped nodes' children were swaped and total time O(n)
max height of stack was the n/2.
Tree was traversed from top to bottom while each time moving left and storing the right children to be treated afterward.
*Total O(log\(_{\text{2}}\) n) + O(n) = O(n)*p

\section{Q2/B Time complexity of spiral order travesal}
\label{sec:org30ab638}
ans: \textbf{O(n)+O(n) = O(n)}
\subsection{Detail}
\label{sec:org052ed6a}
   Two Stacks were used. 'left\(_{\text{to}}\)\(_{\text{right}}\)' and 'right\(_{\text{to}}\)\(_{\text{left}}\)' traversal was made level vise to push all elements from left to right and then right to left.
O(n) was for traversal as each node was visited and O(n) was for pushing and then poping each element.

\section{Q4/B Time complexity of flattening a Binary Tree}
\label{sec:org41ce498}
ans: \textbf{O(n)}
\subsection{Detail}
\label{sec:orgc2596e9}
each node was visited recursively preorder and that node was inserted in \textbf{linkList} ojbect. Note: Each node was inserted instead of data which was more difficult but it gives the real picture of how the tree was flattened.
Therefore my flattened list looks like
\section{Q5/B Time complexity of joining Two AVL Trees}
\label{sec:org582c1a5}
ans: \textbf{O(h\(_{\text{1}}\)+h\(_{\text{2}}\))} where h\(_{\text{1}}\) is the hight of T\(_{\text{1}}\) and h\(_{\text{2}}\) is the height of T\(_{\text{2}}\)
Note: T\(_{\text{2}}\) is not merged node by node, Whole tree is attached at once almost
\subsection{Detail}
\label{sec:org4c50244}
T\(_{\text{2}}\)'s smallest Element was broken and made root of new tree T\(_{\text{3}}\).
Then T\(_{\text{1}}\) was traversed right until height becomes = h\(_{\text{2}}\)
Then T\(_{\text{1}}\) was broken in two parts at that that root and was connected to left of T\(_{\text{3}}\)
T\(_{\text{2}}\) was attached on the right side of T\(_{\text{3}}\)
Then that T\(_{\text{3}}\) was attached to parent of broken node in T\(_{\text{1}}\)
Node breaking is constant and this algorithm is very efficient for AVL trees with greater hieghts.
O(h\(_{\text{1}}\)+h\(_{\text{2}}\)) is just due to traversal for the appropriate place where one T\(_{\text{2}}\) could be attached.
\end{document}